\documentclass{article}
\usepackage[utf8]{inputenc}
\usepackage{amsmath}
\usepackage{xcolor}
\usepackage{natbib}
\usepackage{appendix}
\usepackage{graphicx}
\graphicspath{ {../Graph/} }

\setlength{\parskip}{\baselineskip}%

\title{Rationale of "EM"
procedure in estimation of nonlinear discrete choice model with linear index}
\date{\vspace{-5ex}}


\begin{document}

\appendix
\section{Income Distribution Fitting}

The data includes 7895 census tracts in California. For each tract, we observe the $20\%$, $40\%$, $60\%$,  $80\%$ and $95\%$ percentiles of its annual income distribution\footnote{In the dataest, some percentiles are top-coeded as "$2,500-$" or "250,000+" due to the concern of personal information privacy. Since all tracts have more than 2 observed percentiles or within-percentile means besides the top-coded quantities, they are ignored in the fitting procedure innocuously from the perspective of identification.}, as well as the within-percentile means.  A unique log normal distribution is fitted for each tract by minimizing the distance between the observed quantities and the model predictions, weighted by margins of error i.e.


$$(\mu, \sigma) = \mbox{argmin} \sum_{i = 1}^5 \big(q_i - F^{-1}(p_i; \mu, \sigma^2)\big)^2/w^2_{q_i} + \sum_{i = 0}^5(e_i - E[x| F(x; \mu, \sigma^2) \in (p_{i}, p_{i + 1})])^2/w^2_{e_i}$$


where 
\begin{itemize}
	\item $F(x; \mu, \sigma^2)$ is the cumulative density function (CDF) of a log-normal distribution parameterized by $\mu$ and $\sigma^2$, i.e., $log(X) \sim \mathcal{N}(\mu, \sigma^2)$, so $F^{-1}(p; \mu, \sigma^2)$ is the inverse CDF.
	\item $P = \{p_0, \ p_1, \ p_2, \ p_3, \ p_4, \ p_5, \ p_6\} = \{0, \ 0.2, \ 0.4, \ 0.6, \ 0.8, \ 0.95, \ 1\}$
	\item $q_i$  are observed percentiles at $p_i$,.
	\item $e_i$ are observed average income within the percentile $p_i$ and $p_{i + 1}$.
	\item $w_{q_i}$ and $w_{e_i}$ are the margins of error, normalized by the sum of their squares.
\end{itemize}


% latex table generated in R 3.3.0 by xtable 1.8-3 package
% Wed Oct 24 19:57:14 2018
\begin{table}[ht]
\centering
\caption{Tracts with Mininum and Maximum Income Mean \& SD (US\$)} 
\begin{tabular}{clrr}
  \hline
 & Tract.ID & Mean.of.Annual.Income & SD.of.Annual.Income \\ 
  \hline
Mininum Mean & 06037206300 & 11986.05 & 10797.56 \\ 
  Maximum Mean & 06081611400 & 424735.10 & 610402.98 \\ 
  Minimum SD & 06075017801 & 13928.19 & 4979.93 \\ 
  Maximum SD & 06081613400 & 417421.95 & 797969.61 \\ 
   \hline
\end{tabular}
\label{ext.case.list}
\end{table}

Denote the fitted parameters as $(\hat{\mu}, \hat{\sigma}^2)$, then the mean and variance of income is calculated as $E[x] = e^{\hat{\mu} +\frac{\hat{\sigma}^2}{2}}$, $Var[x] = e^{2\hat{\mu} +\hat{\sigma}^2}\sqrt{e^{\frac{\hat{\sigma}}{2} } - 1}$. Table.\ref{ext.case.list} is a list of tracts with the highest and lowest average income and most and least dispersed income.


\begin{figure}[h]
	\centering
	\includegraphics[scale=0.5]{"CA Income Distribution_q&ex_w"}
	\caption{Income Distribution in California}
	\label{inc.dist}
\end{figure}


We also simulate the income distribution of the entire California by taking random draws from the fitted distribution of each tract weighted by its population. The mean is $\$92074.96$, and the standard deviation is $\$121784.6$. Figure.\ref{inc.dist} summarizes this aggregate income distribution as well as those extreme cases.





\end{document}